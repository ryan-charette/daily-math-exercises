\documentclass{article}
\usepackage{amsthm}
\usepackage{amsmath}
\usepackage{amssymb}
\usepackage{enumerate}
\usepackage{mathrsfs}
\usepackage{mathtools}

\theoremstyle{definition}
\newtheorem{exercise}{Exercise}
\newtheorem*{solution}{Solution}

\DeclareMathOperator{\N}{\mathbb{N}}
\DeclareMathOperator{\Z}{\mathbb{Z}}
\DeclareMathOperator{\R}{\mathbb{R}}
\DeclareMathOperator{\C}{\mathbb{C}}

\begin{document}

\section*{Week 1}
\subsection*{Day 1}

\begin{exercise}
    If \(r\) is rational (\(r\neq0\)) and \(x\) is irrational, prove that \(r+x\) and \(rx\) are irrational.
\end{exercise}
\begin{solution}
    Since \(r\) is rational, we can write \(r=\frac{c}{d}\) for some nonzero \(c,d\in\Z\).
    \begin{enumerate}[(i)]
        \item Suppose \(r+x\) can be written as \(\frac{a}{b}\) for \(a,b\in\Z\), \(b\neq0\). Then 
        \[x=\frac{a}{b}-r=\frac{a}{b}-\frac{c}{d}=\frac{ad-bc}{bd}.\]
        \item Suppose \(rx\) can be written as \(\frac{a}{b}\) for \(a,b\in\Z\), \(b\neq0\). Then
        \[x=\frac{a}{b}r^{-1}=\frac{a}{b}\cdot\frac{d}{c}=\frac{ad}{bc}.\]
    \end{enumerate}
    In either case, we have a contradiction to the irrationality of \(x\).
\end{solution}

\begin{exercise}
    Prove that for any \(m,n\in\N\), \(m^{1/n}\) is either irrational or an integer.
\end{exercise}
\begin{solution}
    Suppose to the contrary that \(m^{1/n}=\frac{a}{b}\) for some \(a,b\in\N\) with \(b\neq1\). Without loss of generality, let \(a\) and \(b\) be coprime. Then by the fundamental theorem of arithmetic, \(b^n\nmid a^n\). However, \(m=(m^{1/n})^n=a^n/b^n\), implying that \(m\not\in\Z\), a contradiction.
\end{solution}

\begin{exercise}
    Let \(\mathbb{F}\) be a field and \(x,y,z\in\mathbb{F}\), \(x\neq0\). Using the field axioms for multiplication, prove the following statements
    \begin{enumerate}[(a)]
        \item If \(xy=xz\), then \(y=z\).
        \item If \(xy=x\), then \(y=1\).
        \item If \(xy=1\), then \(y=x^{-1}\).
        \item \((x^{-1})^{-1}=x\).
    \end{enumerate}
\end{exercise}
\begin{solution}
    If \(xy=xz\), the axioms for multiplication give
    \[y=1y=(x^{-1}x)y=x^{-1}(xy)=x^{-1}(xz)=(x^{-1}x)z=1z=z.\]
    This proves (a). Take \(z=1\) in (a) to obtain (b). Take \(z=x^{-1}\) in (a) to obtain (c). Since \(x^{-1}x=1\), (c) (with \(x^{-1}\) in place of \(x\)) gives (d).
\end{solution}

\begin{exercise}
    Prove that if \(\sim\) is an equivalence relation on a nonempty set \(S\), then the corresponding family \(\mathscr{P}_\sim\) is a partition of \(S\).
\end{exercise}
\begin{solution}
    Let \(a\in S\). By reflexivity, \(a\in[a]_\sim\). Each element of \(S\) is contained in a nonempty equivalence class; hence the union over all such equivalence classes is \(S\). To show that the equivalence classes are mutually disjoint, consider some \(b\in S\) such that \(a\) is not equivalent to \(b\). Assume that there exists some \(x\in [a]_\sim\cap[b]_\sim\). Then \(a\sim x\) and \(b\sim x\), hence \(x\sim b\) by symmetry and \(a\sim b\) by transitivity. This is a contradiction, so \([a]_\sim\cap[b]_\sim=\emptyset\). The elements of \(\mathscr{P}_\sim\) are nonempty, disjoint, and their union is \(S\); thus \(\mathscr{P}_\sim\) is a partition of \(S\).
\end{solution}

\begin{exercise}
    Let \(X\) be a topological space; let \(A\) be a subset of \(X\). Suppose that for each \(x\in A\) there is an open set \(U\) containing \(x\) such that \(U\subset A\). Show that \(A\) is open in \(X\).
\end{exercise}
\begin{solution}
    Associate to each \(x\in A\) an open set \(U_x\subset A\). Then
    \[V=\bigcup_{x\in A} U_x\]
    is a union of open sets; hence \(V\) is open. Since each \(U_x\) is a subset of \(A\), we have \(V\subset A\). However, each \(x\in A\) is contained in some \(U_x\) (and therefore in \(V\)) so we also have that \(A\subset V\). Thus \(A=V\), implying that \(A\) is open.
\end{solution}

\subsection*{Day 2}

\begin{exercise}
    Show that the system of all matrices of the special form
    \[\begin{pmatrix*}[r]
        \alpha & \beta \\
        -\beta & \alpha
    \end{pmatrix*},\]
    combined by matrix addition and matrix multiplication, is isomorphic to the field of complex numbers.
\end{exercise}
\begin{solution}
    Let \(z=(a,b)\) and \(w=(c,d)\) be two complex numbers. Represent \(z\) and \(w\) by the matrices
    \[Z=
    \begin{pmatrix*}[r]
        a & b \\
        -b & a
    \end{pmatrix*}
    \quad\text{and}\quad
    W=
    \begin{pmatrix*}[r]
        c & d \\
        -d & c
    \end{pmatrix*}.\]
    Then the sum \(Z+W\) and the product \(ZW\) are given by
    \[Z+W=
    \begin{pmatrix*}[r]
        a+c & b+d \\
        -b-d & a+c
    \end{pmatrix*}
    \quad\text{and}\quad
    ZW=
    \begin{pmatrix*}[r]
        ac-bd & ad+bc \\
        -ad-bc & ac-bd
    \end{pmatrix*}.\]
    Therefore \(Z+W\) corresponds to the complex number \((a+c, b+d)\) and \(ZW\) corresponds to \((ac-bd, ad+bc)\). These are exactly the values \(z+w\) and \(zw\).
\end{solution}

\begin{exercise}
    Show that the complex-number system can be thought of as the field of all polynomials with real coefficients modulo the irreducible polynomial \(x^2+1\).
\end{exercise}
\begin{solution}
    Define a map \(\phi:\R[x]/(x^2+1)\to\C\) by \(\phi(a+bx)=a+bi\). The map \(\phi\) is a homomorphism:
    \begin{align*}
        \phi((a+bx)+(c+dx))&=\phi((a+c)+(b+d)x)\\
        &=(a+c)+(b+d)i \\
        &=(a+bi)+(c+di) \\
        &=\phi(a+bx)+\phi(c+dx) \\
        &\\
        \phi((a+bx)(c+dx))&=\phi(ac+(ad+bc)x+bdx^2) \\
        &=\phi(ac+(ad+bc)x+bd(-1)) \\
        &=\phi((ac-bd)+(ad+bc)x) \\
        &=(ac-bd)+(ad+bc)i=(a+bi)(c+di) \\
        &=\phi(a+bx)\phi(c+dx).
    \end{align*}
    It is also bijective: if \(\phi(a+bx)=\phi(c+dx)\), then \(a+bi=c+di\), implying \(a=c\) and \(b=d\). Thus, \(\phi\) is injective. Every complex number \(a+bi\) is the image of \(a+bx\) under \(\phi\), so \(\phi\) is surjective. Therefore, \(\phi\) is an isomorphism of rings and we conclude that \(\C\cong\R[x]/(x^2+1)\).
\end{solution}

\begin{exercise}
    Let \(g:\R\to\R\) be defined by
    \[g(x)=\int_0^x f(t)\,dt=\int_0^x t^{1/3}\,dt=\frac{3}{4}x^{4/3}.\]
    Show that the function \(h(x)=\int_0^x g(t)\,dt\) is \(C^2\) but not \(C^3\) at \(x=0\).
\end{exercise}
\begin{solution}
    The second derivative \(h''(x)=g'(x)=f(x)=x^{1/3}\) is continuous everywhere on its domain, so \(h(x)\) is \(C^2\). The third derivative \(h'''(x)=\frac{1}{3}x^{-2/3}\) has a discontinuity (a vertical asymptote) at \(x=0\); that is, \(h\) is not \(C^3\) at \(x=0\).
\end{solution}

\begin{exercise}
    Show that the function \(f:(-\pi/2,\pi/2)\to\R\), \(f(x)=\tan x\), is a diffeomorphism.
\end{exercise}
\begin{solution}
    First we show that \(f\) is a bijection. A strictly increasing function is injective because it never takes the same value twice. The function \(f\) strictly increasing on its domain since \(f'(x)=\sec^2 x>0\) for all \(x\in(-\pi/2,\pi/2)\); hence \(f\) is injective. Moreover, 
    \[\lim_{x\to-\pi/2^+}=-\infty\quad\text{and}\quad\lim_{x\to\pi/2^-}=\infty.\]
    Therefore, \(f\) maps \((-\pi/2,\pi/2)\) onto the entire real line \(\R\); that is, \(f\) is surjective. 
    
    The function \(f\) is smooth by induction; \(f'(x)=\sec^2 x\) and \(f''(x)=2\tan x\sec^2 x\), so for \(n>2\), \(f^{(n)}\) can be written as the sum of products of differentiable functions, and is therefore itself differentiable. The inverse function \(f^{-1}(y)=\arctan y\) has first derivative \((1+y^2)^{-1}\) with respect to \(y\), which is the reciprocal of a polynomial. For all \(y\in\R\), \(1+y^2\geq1>0\), ensuring that the derivative is well-defined for all real numbers. Thus, \(f^{-1}(y)\) inherits smoothness from the smoothness of polynomials.

    The function \(f\) is a bijective \(C^\infty\) map with \(C^\infty\) inverse, therefore \(f\) satisfies the definition of a diffeomorphism.
\end{solution}

\end{document}
