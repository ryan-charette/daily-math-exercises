\documentclass{article}
\usepackage{amsthm}
\usepackage{amsmath}
\usepackage{amssymb}
\usepackage{enumerate}

\theoremstyle{definition}
\newtheorem{exercise}[theorem]{Exercise}

\DeclareMathOperator{\N}{\mathbb{N}}
\DeclareMathOperator{\Z}{\mathbb{Z}}

\begin{document}

\section*{Analysis}
\subsection*{The Real and Complex Numbers}

\begin{exercise}
    If \(r\) is rational (\(r\neq0\)) and \(x\) is irrational, prove that \(r+x\) and \(rx\) are irrational.
\end{exercise}
\begin{quote}
    Since \(r\) is rational, we can write \(r=\frac{c}{d}\) for some nonzero \(c,d\in\Z\).
    \begin{enumerate}[(i)]
        \item Suppose \(r+x\) can be written as \(\frac{a}{b}\) for \(a,b\in\Z\), \(b\neq0\). Then 
        \[x=\frac{a}{b}-r=\frac{a}{b}-\frac{c}{d}=\frac{ad-bc}{bd}.\]
        \item Suppose \(rx\) can be written as \(\frac{a}{b}\) for \(a,b\in\Z\), \(b\neq0\). Then
        \[x=\frac{a}{b}r^{-1}=\frac{a}{b}\cdot\frac{d}{c}=\frac{ad}{bc}.\]
    \end{enumerate}
    In either case, we have a contradiction to the irrationality of \(x\).
\end{quote}

\end{document}
