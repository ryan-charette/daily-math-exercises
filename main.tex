\documentclass{article}
\usepackage{amsthm}
\usepackage{amsmath}
\usepackage{amssymb}
\usepackage{mathrsfs}
\usepackage{enumerate}

\theoremstyle{definition}
\newtheorem{exercise}{Exercise}
\newtheorem*{solution}{Solution}

\DeclareMathOperator{\N}{\mathbb{N}}
\DeclareMathOperator{\Z}{\mathbb{Z}}

\begin{document}

\section*{Analysis}
\subsection*{The Real and Complex Numbers}

\begin{exercise}
    If \(r\) is rational (\(r\neq0\)) and \(x\) is irrational, prove that \(r+x\) and \(rx\) are irrational.
\end{exercise}
\begin{solution}
    Since \(r\) is rational, we can write \(r=\frac{c}{d}\) for some nonzero \(c,d\in\Z\).
    \begin{enumerate}[(i)]
        \item Suppose \(r+x\) can be written as \(\frac{a}{b}\) for \(a,b\in\Z\), \(b\neq0\). Then 
        \[x=\frac{a}{b}-r=\frac{a}{b}-\frac{c}{d}=\frac{ad-bc}{bd}.\]
        \item Suppose \(rx\) can be written as \(\frac{a}{b}\) for \(a,b\in\Z\), \(b\neq0\). Then
        \[x=\frac{a}{b}r^{-1}=\frac{a}{b}\cdot\frac{d}{c}=\frac{ad}{bc}.\]
    \end{enumerate}
    In either case, we have a contradiction to the irrationality of \(x\).
\end{solution}

\begin{exercise}
    Prove that for any \(m,n\in\N\), \(m^{1/n}\) is either irrational or an integer.
\end{exercise}
\begin{solution}
    Suppose to the contrary that \(m^{1/n}=\frac{a}{b}\) for some \(a,b\in\N\) with \(b\neq1\). Without loss of generality, let \(a\) and \(b\) be coprime. Then by the fundamental theorem of arithmetic, \(b^n\nmid a^n\). However, \(m=(m^{1/n})^n=a^n/b^n\), implying that \(m\not\in\Z\), a contradiction.
\end{solution}

\begin{exercise}
    Let \(\mathbb{F}\) be a field and \(x,y,z\in\mathbb{F}\), \(x\neq0\). Using the field axioms for multiplication, prove the following statements
    \begin{enumerate}[(a)]
        \item If \(xy=xz\), then \(y=z\).
        \item If \(xy=x\), then \(y=1\).
        \item If \(xy=1\), then \(y=x^{-1}\).
        \item \((x^{-1})^{-1}=x\).
    \end{enumerate}
\end{exercise}
\begin{solution}
    If \(xy=xz\), the axioms for multiplication give
    \[y=1y=(x^{-1}x)y=x(x^{-1}y)=x^{-1}(xz)=(x^{-1}x)z=1z=z.\]
    This proves (a). Take \(z=1\) in (a) to obtain (b). Take \(z=x^{-1}\) in (a) to obtain (c). Since \(x^{-1}x=1\), (c) (with \(x^{-1}\) in place of \(x\)) gives (d).
\end{solution}

\section*{Algebra}
\subsection*{Sets and Categories}

\begin{exercise}
    Prove that if \(\sim\) is an equivalence relation on a nonempty set \(S\), then the corresponding family \(\mathscr{P}_\sim\) is a partition of \(S\).
\end{exercise}
\begin{solution}
    Let \(a\in S\). By reflexivity, \(a\in[a]_\sim\). Each element of \(S\) is contained in a nonempty equivalence class; hence the union over all such equivalence classes is \(S\). To show that the equivalence classes are mutually disjoint, consider some \(b\in S\) such that \(a\) is not equivalent to \(b\). Assume that there exists some \(x\in [a]_\sim\cap[b]_\sim\). Then \(a\sim x\) and \(b\sim x\), hence \(x\sim b\) by symmetry and \(a\sim b\) by transitivity. This is a contradiction, so \([a]_\sim\cap[b]_\sim=\emptyset\). The elements of \(\mathscr{P}_\sim\) are nonempty, disjoint, and their union is \(S\); thus \(\mathscr{P}_\sim\) is a partition of \(S\).
\end{solution}

\section*{Topology}
\subsection*{Topological Spaces}

\begin{exercise}
    Let \(X\) be a topological space; let \(A\) be a subset of \(X\). Suppose that for each \(x\in A\) there is an open set \(U\) containing \(x\) such that \(U\subset A\). Show that \(A\) is open in \(X\).
\end{exercise}
\begin{solution}
    Associate to each \(x\in A\) an open set \(U_x\subset A\). Then
    \[V=\bigcup_{x\in A} U_x\]
    is a union of open sets; hence \(V\) is open. Since each \(U_x\) is a subset of \(A\), we have \(V\subset A\). However, each \(x\in A\) is contained in some \(U_x\) (and therefore in \(V\)) so we also have that \(A\subset V\). Thus \(A=V\), implying that \(A\) is open.
\end{solution}

\end{document}
