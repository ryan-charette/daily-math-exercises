\documentclass{article}
\pagestyle{empty}
\usepackage{amsmath}
\usepackage{amssymb}
\usepackage{parskip}

\DeclareMathOperator{\R}{\mathbb{R}}
\DeclareMathOperator{\C}{\mathbb{C}}

\begin{document}

\textbf{Day 1.} Let \(R\) be a ring and let \((a,b)\), \((c,d)\), and \((e,f)\) be elements in \(R\times R\). To show that \(R\times R\) with the given operations forms a ring, we need to verify that the set together with these operations satisfies the ring axioms. Since addition in \(R\) is associative and commutative, the same holds for \(R\times R\). The additive identity is \((0,0)\), and the additive inverse of \((a,b)\) is \((-a,-b)\). As for multiplication,
\begin{align*}
    [(a,b)(c,d)](e,f) &= (ac-bd, ad+bc)(e,f) \\
    &= ((ac-bd)e-(ad+bc)f, (ac-bd)f+(ad+bc)e) \\
    &= (ace-bde-adf-bcf, acf-bdf+ade+bce),
\end{align*}
and, similarly,
\begin{align*}
    (a,b)[(c,d)(e,f)] &= (a,b)(ce-df,cf+de) \\
    &= (a(ce-df)-b(cf+de), a(cf+de)+b(ce-df)) \\
    &= (ace-adf-bcf-bde, acf+ade+bce-bdf).
\end{align*}
Therefore, \([(a,b)(c,d)](e,f)=(a,b)[(c,d)(e,f)]\); in other words, multiplication is associative. The multiplicative identity is \((1,0)\), because
\[(a,b)(1,0)=(a1-b0, a0+b1)=(a,b)\]
and
\[(1,0)(a,b)=(1a-0b, 1b+0a)=(a,b).\]
All that remains to be shown are the distributive laws:
\begin{align*}
    (a,b)[(c,d)+(e,f)] &= (a,b)(c+e,d+f) \\
    &= (a(c+e)-b(d+f),a(d+f)+b(c+e)) \\
    &= (ac+ae-bd-bf, ad+af+bc+be) \\
    &= ((ac-bd)+(ae-bf), (ad+bc)+(af+be)) \\
    &= (ac-bd,ad+bc)+(ae-bf,af+be) \\
    &= (a,b)(c,d)+(a,b)(e,f)
\end{align*}
and
\begin{align*}
    [(a,b)+(c,d)](e,f) &= (a+c,b+d)(e,f) \\
    &= ((a+c)e-(b+d)f, (a+c)f+(b+d)e) \\
    &= (ae-bf+ce-df, af+be+cf+de) \\
    &= ((ae-bf)+(ce-df), (af+be)+(cf+de)) \\
    &= (ae-bf, af+be) + (ce-df, cf+de) \\
    &= (a,b)(e,f) + (c,d)(e,f).
\end{align*}
With these operations, \(R\times R\) is a ring. When \(R=\R\), the ring \(R\times R\) with these operations is \textit{isomorphic} to the ring of complex numbers \(\C\). The 
elements \((a,b)\) correspond to the complex numbers \(a+bi\), and the operations are equivalent to complex addition and multiplication. 

\end{document}
