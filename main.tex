\documentclass{article}
\usepackage{amsthm}
\usepackage{amsmath}
\usepackage{amssymb}
\usepackage{enumerate}

\theoremstyle{definition}
\newtheorem{exercise}[theorem]{Exercise}

\DeclareMathOperator{\N}{\mathbb{N}}
\DeclareMathOperator{\Z}{\mathbb{Z}}

\begin{document}

\section*{Analysis}
\subsection*{The Real and Complex Numbers}

\begin{exercise}
    If \(r\) is rational (\(r\neq0\)) and \(x\) is irrational, prove that \(r+x\) and \(rx\) are irrational.
\end{exercise}
\begin{quote}
    Since \(r\) is rational, we can write \(r=\frac{c}{d}\) for some nonzero \(c,d\in\Z\).
    \begin{enumerate}[(i)]
        \item Suppose \(r+x\) can be written as \(\frac{a}{b}\) for \(a,b\in\Z\), \(b\neq0\). Then 
        \[x=\frac{a}{b}-r=\frac{a}{b}-\frac{c}{d}=\frac{ad-bc}{bd}.\]
        \item Suppose \(rx\) can be written as \(\frac{a}{b}\) for \(a,b\in\Z\), \(b\neq0\). Then
        \[x=\frac{a}{b}r^{-1}=\frac{a}{b}\cdot\frac{d}{c}=\frac{ad}{bc}.\]
    \end{enumerate}
    In either case, we have a contradiction to the irrationality of \(x\).
\end{quote}

\begin{exercise}
    Prove that there is no rational number whose square is 12.
\end{exercise}
\begin{quote}
    We can prove the more general statement that for any \(m,n\in\N\), \(m^{1/n}\) is either irrational or an integer. Suppose to the contrary that \(m^{1/n}=\frac{a}{b}\) for some \(a,b\in\N\) with \(b\neq1\). Without loss of generality, let \(a\) and \(b\) be coprime. Then by the fundamental theorem of arithmetic, \(b^n\nmid a^n\). However, \(m=(m^{1/n})^n=a^n/b^n\), implying that \(m\not\in\Z\), a contradiction.    
\end{quote}

\begin{exercise}
    Let \(\mathbb{F}\) be a field and \(x,y,z\in\mathbb{F}\), \(x\neq0\). Using the field axioms for multiplication, prove the following statements
    \begin{enumerate}[(a)]
        \item If \(xy=xz\), then \(y=z\).
        \item If \(xy=x\), then \(y=1\).
        \item If \(xy=1\), then \(y=x^{-1}\).
        \item \((x^{-1})^{-1}=x\).
    \end{enumerate}
\end{exercise}
\begin{quote}
    \begin{enumerate}[(a)]
        \item \(y=1y=(x^{-1}x)y=x(x^{-1}y)=x^{-1}(xz)=(x^{-1}x)z=1z=z.\)
        \item Let \(z=1\). Then \(xz=x1=x=xy\), so (a) implies \(y=1\).
        \item Let \(z=x^{-1}\). Then \(xz=xx^{-1}=1=xy\), so
        (a) implies \(y=x^{-1}\).
        \item Since \(x^{-1}x=1\), (c) implies \(x=(x^{-1})^{-1}\).       
    \end{enumerate}
\end{quote}
\end{document}
