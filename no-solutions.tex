\documentclass{article}
\usepackage{amsthm}
\usepackage{amsmath}
\usepackage{amssymb}
\usepackage{enumerate}
\usepackage[mathcal]{euscript}
\usepackage{mathrsfs}
\usepackage{mathtools}

\theoremstyle{definition}
\newtheorem*{exercise}{Exercise}
\newtheorem*{solution}{Solution}

\DeclareMathOperator{\N}{\mathbb{N}}
\DeclareMathOperator{\Z}{\mathbb{Z}}
\DeclareMathOperator{\R}{\mathbb{R}}
\DeclareMathOperator{\C}{\mathbb{C}}

\renewcommand{\Re}{\operatorname{Re}}
\renewcommand{\Im}{\operatorname{Im}}

\begin{document}

%%%%%%%%%%%%%%%%%%%%%%%%%%%%%%%%%%%%%%%%%%%%%%%%%%%%%%%%%%%%%%%%%%%%%%%%%%%%%%%%%%%%%%%%%%%%%%%%%%%%%%%%%%%%%%%%%%%%%%%%%%%%%%%%%%%%%%%%
%%                                            Exercise Difficulty Rating Guide                                                        %%
%%%%%%%%%%%%%%%%%%%%%%%%%%%%%%%%%%%%%%%%%%%%%%%%%%%%%%%%%%%%%%%%%%%%%%%%%%%%%%%%%%%%%%%%%%%%%%%%%%%%%%%%%%%%%%%%%%%%%%%%%%%%%%%%%%%%%%%%
%% Trival: The proof follows directly from definitions or previously established results with no additional reasoning required.       %%
%% Elementary: The proof involves basic logical steps, often relying on straightforward algebra, arithmetic, or fundamental theorems. %%
%% Intermediate: The proof requires combining multiple theorems or concepts, with moderate problem-solving and reasoning.             %%
%% Advanced: The proof demands creative reasoning and deep understanding of important concepts.                                       %%
%% Challenging: The proof is highly complex, requiring extensive background knowledge, mastery of advanced techniques.                %%
%%%%%%%%%%%%%%%%%%%%%%%%%%%%%%%%%%%%%%%%%%%%%%%%%%%%%%%%%%%%%%%%%%%%%%%%%%%%%%%%%%%%%%%%%%%%%%%%%%%%%%%%%%%%%%%%%%%%%%%%%%%%%%%%%%%%%%%%

\section*{Week 1: Rings}

\subsection*{Day 1}
\begin{exercise}
    % Trivial
    Let \(a\) and \(b\) be elements of a ring \(R\). Prove that the equation \(a+x=b\) has a unique solution in \(R\).
\end{exercise}

\subsection*{Day 2}
\begin{exercise}
    % Trivial
    If \(R\) is a ring with identity and \(\phi:R\to S\) is a homomorphism from \(R\) to a ring \(S\), prove that \(\phi(1_R)\) is idempotent in \(S\).
\end{exercise}

\subsection*{Day 3}
\begin{exercise}
    % Elementary
    Show that \(\Z[i]\) is a subring of \(\C\).
\end{exercise}

\subsection*{Day 4}
\begin{exercise}
    % Elementary
    Let \(S\) and \(T\) be subrings of a ring \(R\). Prove that \(S\cap T\) is a subring of \(R\).
\end{exercise}

\subsection*{Day 5}
\begin{exercise}
    % Intermediate
    Prove that the only idempotents in an integral domain \(R\) are 0 and 1.
\end{exercise}

\section*{Week 2: Fields}

\subsection*{Day 1}
\begin{exercise}
    % Trivial
    Let \(\mathbb{F}\) be a field and \(x,y,z\in\mathbb{F}\), \(x\neq0\). Using the field axioms for multiplication, prove the following statements
    \begin{enumerate}[(a)]
        \item If \(xy=xz\), then \(y=z\).
        \item If \(xy=x\), then \(y=1\).
        \item If \(xy=1\), then \(y=x^{-1}\).
        \item \((x^{-1})^{-1}=x\).
    \end{enumerate}
\end{exercise}

\subsection*{Day 2}
\begin{exercise}
    % Intermediate
    Show that the system of all matrices of the form
    \[\begin{pmatrix*}[r]
        \alpha & \beta \\
        -\beta & \alpha
    \end{pmatrix*}\]
    combined by matrix addition and matrix multiplication is isomorphic to the field of complex numbers.
\end{exercise}

\subsection*{Day 3}
\begin{exercise}
    % Advanced
    Show that the complex-number system can be thought of as the field of all polynomials with real coefficients modulo the irreducible polynomial \(x^2+1\).
\end{exercise}

\section*{Week 3: Smooth Functions}

\subsection*{Day 1}
\begin{exercise}
    % Intermediate
    Let \(g:\R\to\R\) be defined by
    \[g(x)=\int_0^x f(t)\,dt=\int_0^x t^{1/3}\,dt=\frac{3}{4}x^{4/3}.\]
    Show that the function \(h(x)=\int_0^x g(t)\,dt\) is \(C^2\) but not \(C^3\) at \(x=0\).
\end{exercise}

\subsection*{Day 2}
\begin{exercise}
    % Intermediate
    Find a bijective \(C^\infty\) map \(f:\R\to\R\) that does not have a \(C^\infty\) inverse.
\end{exercise}

\subsection*{Day 3}
\begin{exercise}
    % Advanced
    Show that the function \(f:(-\pi/2,\pi/2)\to\R\), \(f(x)=\tan x\), is a diffeomorphism.
\end{exercise}

\section*{Week 4: Number systems}

\subsection*{Day 1}
\begin{exercise}
    % Elementary
    If \(r\) is rational (\(r\neq0\)) and \(x\) is irrational, prove that \(r+x\) and \(rx\) are irrational.
\end{exercise}

\subsection*{Day 2}
\begin{exercise}
    % Elementary
    Prove that for any natural numbers \(a,b,c\), we have \((a+b)+c=a+(b+c)\).
\end{exercise}

\subsection*{Day 3}
\begin{exercise}
    % Intermediate
    Let \(A\) be a nonempty set of real numbers which is bounded below. Let \(-A\) be the set of all numbers \(-x\), where \(x\in A\). Prove that \(\inf A=-\sup(-A)\).
\end{exercise}

\subsection*{Day 4}
\begin{exercise}
    % Intermediate
    Prove that for any \(m,n\in\N\), \(m^{1/n}\) is either irrational or an integer.
\end{exercise}

\subsection*{Day 5}
\begin{exercise}
    % Intermediate
    Find the conditions under which the equation \(az+b\bar{z}+c=0\) in one complex variable has exactly one solution, and compute that solution.
\end{exercise}

\section{Week 5: Sets}

\subsection*{Day 1}
\begin{exercise}
    % Trivial
    Let \(E\) be a nonempty subset of an ordered set; suppose \(\alpha\) is a lower bound of \(E\) and \(\beta\) is an upper bound of \(E\). Prove that \(\alpha\leq\beta\).
\end{exercise}

\subsection*{Day 2}
\begin{exercise}
    % Trivial
    Let \(X\) be a topological space; let \(A\) be a subset of \(X\). Suppose that for each \(x\in A\) there is an open set \(U\) containing \(x\) such that \(U\subset A\). Show that \(A\) is open in \(X\).
\end{exercise}

\subsection*{Day 3}
\begin{exercise}
    % Elementary
    Let \(\mathcal{M}\) be an infinite \(\sigma\)-algebra. Show that \(\mathcal{M}\) contains an infinite sequence of disjoint sets.
\end{exercise}

\subsection*{Day 4}
\begin{exercise}
    % Elementary
    Prove that if \(\sim\) is an equivalence relation on a set \(S\), then the corresponding family \(\mathscr{P}_\sim\) is a partition of \(S\).
\end{exercise}

\end{document}
