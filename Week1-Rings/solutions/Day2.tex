\documentclass{article}
\usepackage{parskip}

\begin{document}
To show that \(N\) forms a subring of \(R\), we need to verify that \(N\) is nonempty, \(N\) is closed under addition and multiplication, and \(N\) has additive inverses. The zero element is nilpotent since \(0^1=0\), so \(N\) is non-empty. For any \(a,b\in N\), we must show that \(a+b\) is also nilpotent. Since \(a\) and \(b\) are nilpotent, there exist positive integers \(m\) and \(n\) such that \(a^m=0\) and \(b^n=0\). Using the binomial theorem,
\[(a+b)^k=\sum_{i=0}^k {k \choose i} a^ib^{k-i}.\]
For each term \(a^i b^{k-i}\), if \(i\geq m\), then \(a^i=0\) because \(a^m=0\), and if \(k-i\geq n\), then \(b^{k-i}=0\) because \(b^n=0\). Set \(k=m+n-1\) so that when \(i<m\),
\[k-i=(m+n-1)-i= n+(m-i)-1\geq n+1-1 = n.\]
Then all terms in the expansion become zero, and \((a+b)^k=0\). Hence \(a+b\) is nilpotent. We must also show that \(ab\) is nilpotent. Since \(a^m=0\) and \(b^n=0\), 
\[(ab)^{m+n}=a^{m+n}b^{m+n}=a^ma^nb^mb^n=0\cdot a^nb^m\cdot 0=0\]
because any multiple of zero is zero. Thus, \(ab\) is nilpotent. Finally, we must show that \(-a\) is also nilpotent. Since \(a^n=0\),
\[(-a)^n=(-1)^n a^n=(-1)^n\cdot 0=0.\]
Thus, \(-a\) is nilpotent. Since all subring criteria are satisfied, \(N\) is indeed a subring of \(R\).
\end{document}
