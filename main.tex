\documentclass{article}
\usepackage{amsthm}
\usepackage{amsmath}
\usepackage{amssymb}
\usepackage{enumerate}
\usepackage[mathcal]{euscript}
\usepackage{mathrsfs}
\usepackage{mathtools}

\theoremstyle{definition}
\newtheorem{exercise}{Exercise}
\newtheorem*{solution}{Solution}

\DeclareMathOperator{\N}{\mathbb{N}}
\DeclareMathOperator{\Z}{\mathbb{Z}}
\DeclareMathOperator{\R}{\mathbb{R}}
\DeclareMathOperator{\C}{\mathbb{C}}

\renewcommand{\Re}{\operatorname{Re}}
\renewcommand{\Im}{\operatorname{Im}}

\begin{document}

\section*{Week 1}
\subsection*{Day 1}

\begin{exercise}
    If \(r\) is rational (\(r\neq0\)) and \(x\) is irrational, prove that \(r+x\) and \(rx\) are irrational.
\end{exercise}
\begin{solution}
    Since \(r\) is rational, we can write \(r=\frac{c}{d}\) for some nonzero \(c,d\in\Z\).
    \begin{enumerate}[(i)]
        \item Suppose \(r+x\) can be written as \(\frac{a}{b}\) for \(a,b\in\Z\), \(b\neq0\). Then 
        \[x=\frac{a}{b}-r=\frac{a}{b}-\frac{c}{d}=\frac{ad-bc}{bd}.\]
        \item Suppose \(rx\) can be written as \(\frac{a}{b}\) for \(a,b\in\Z\), \(b\neq0\). Then
        \[x=\frac{a}{b}r^{-1}=\frac{a}{b}\cdot\frac{d}{c}=\frac{ad}{bc}.\]
    \end{enumerate}
    In either case, we have a contradiction to the irrationality of \(x\).
\end{solution}

\begin{exercise}
    Prove that for any \(m,n\in\N\), \(m^{1/n}\) is either irrational or an integer.
\end{exercise}
\begin{solution}
    Suppose to the contrary that \(m^{1/n}=\frac{a}{b}\) for some \(a,b\in\N\) with \(b\neq1\). Without loss of generality, let \(a\) and \(b\) be coprime. Then by the unique factorization of natural numbers, \(b^n\nmid a^n\). However, \(m=(m^{1/n})^n=a^n/b^n\), implying that \(m\not\in\Z\), a contradiction.
\end{solution}

\begin{exercise}
    Let \(\mathbb{F}\) be a field and \(x,y,z\in\mathbb{F}\), \(x\neq0\). Using the field axioms for multiplication, prove the following statements
    \begin{enumerate}[(a)]
        \item If \(xy=xz\), then \(y=z\).
        \item If \(xy=x\), then \(y=1\).
        \item If \(xy=1\), then \(y=x^{-1}\).
        \item \((x^{-1})^{-1}=x\).
    \end{enumerate}
\end{exercise}
\begin{solution}
    If \(xy=xz\), the axioms for multiplication give
    \[y=1y=(x^{-1}x)y=x^{-1}(xy)=x^{-1}(xz)=(x^{-1}x)z=1z=z.\]
    This proves (a). Take \(z=1\) in (a) to obtain (b). Take \(z=x^{-1}\) in (a) to obtain (c). Since \(x^{-1}x=1\), (c) (with \(x^{-1}\) in place of \(x\)) gives (d).
\end{solution}

\begin{exercise}
    Prove that if \(\sim\) is an equivalence relation on a set \(S\), then the corresponding family \(\mathscr{P}_\sim\) is a partition of \(S\).
\end{exercise}
\begin{solution}
    Let \(a\in S\). By reflexivity, \(a\in[a]_\sim\). Each element of \(S\) is contained in a nonempty equivalence class; hence the union over all such equivalence classes is \(S\). To show that the equivalence classes are mutually disjoint, consider some \(b\in S\) such that \(a\) is not equivalent to \(b\). Assume that there exists some \(x\in [a]_\sim\cap[b]_\sim\). Then \(a\sim x\) and \(b\sim x\), hence \(x\sim b\) by symmetry and \(a\sim b\) by transitivity. This is a contradiction, so \([a]_\sim\cap[b]_\sim=\emptyset\). The elements of \(\mathscr{P}_\sim\) are nonempty, disjoint, and their union is \(S\); thus \(\mathscr{P}_\sim\) is a partition of \(S\).
\end{solution}

\begin{exercise}
    Let \(X\) be a topological space; let \(A\) be a subset of \(X\). Suppose that for each \(x\in A\) there is an open set \(U\) containing \(x\) such that \(U\subset A\). Show that \(A\) is open in \(X\).
\end{exercise}
\begin{solution}
    Associate to each \(x\in A\) an open set \(U_x\subset A\). Then
    \[V=\bigcup_{x\in A} U_x\]
    is a union of open sets; hence \(V\) is open. Since each \(U_x\) is a subset of \(A\), we have \(V\subset A\). However, each \(x\in A\) is contained in some \(U_x\) (and therefore in \(V\)) so we also have that \(A\subset V\). Thus \(A=V\), implying that \(A\) is open.
\end{solution}

\subsection*{Day 2}

\begin{exercise}
    Show that the system of all matrices of the form
    \[\begin{pmatrix*}[r]
        \alpha & \beta \\
        -\beta & \alpha
    \end{pmatrix*}\]
    combined by matrix addition and matrix multiplication is isomorphic to the field of complex numbers.
\end{exercise}
\begin{solution}
    Let \(z=(a,b)\) and \(w=(c,d)\) be two complex numbers. Represent \(z\) and \(w\) by the matrices
    \[Z=
    \begin{pmatrix*}[r]
        a & b \\
        -b & a
    \end{pmatrix*}
    \quad\text{and}\quad
    W=
    \begin{pmatrix*}[r]
        c & d \\
        -d & c
    \end{pmatrix*}.\]
    Then the sum \(Z+W\) and the product \(ZW\) are given by
    \[Z+W=
    \begin{pmatrix*}[r]
        a+c & b+d \\
        -b-d & a+c
    \end{pmatrix*}
    \quad\text{and}\quad
    ZW=
    \begin{pmatrix*}[r]
        ac-bd & ad+bc \\
        -ad-bc & ac-bd
    \end{pmatrix*}.\]
    Therefore \(Z+W\) corresponds to the complex number \((a+c, b+d)\) and \(ZW\) corresponds to \((ac-bd, ad+bc)\). These are exactly the values \(z+w\) and \(zw\).
\end{solution}

\begin{exercise}
    Show that the complex-number system can be thought of as the field of all polynomials with real coefficients modulo the irreducible polynomial \(x^2+1\).
\end{exercise}
\begin{solution}
    Define a map \(\phi:\R[x]/(x^2+1)\to\C\) by \(\phi(a+bx)=a+bi\). The map \(\phi\) is a homomorphism:
    \begin{align*}
        \phi((a+bx)+(c+dx))&=\phi((a+c)+(b+d)x)\\
        &=(a+c)+(b+d)i \\
        &=(a+bi)+(c+di) \\
        &=\phi(a+bx)+\phi(c+dx) \\
        &\\
        \phi((a+bx)(c+dx))&=\phi(ac+(ad+bc)x+bdx^2) \\
        &=\phi(ac+(ad+bc)x+bd(-1)) \\
        &=\phi((ac-bd)+(ad+bc)x) \\
        &=(ac-bd)+(ad+bc)i=(a+bi)(c+di) \\
        &=\phi(a+bx)\phi(c+dx).
    \end{align*}
    It is also bijective: if \(\phi(a+bx)=\phi(c+dx)\), then \(a+bi=c+di\), implying \(a=c\) and \(b=d\). Thus, \(\phi\) is injective. Every complex number \(a+bi\) is the image of \(a+bx\) under \(\phi\), so \(\phi\) is surjective. Therefore, \(\phi\) is an isomorphism of rings and we conclude that \(\C\cong\R[x]/(x^2+1)\).
\end{solution}

\begin{exercise}
    Let \(g:\R\to\R\) be defined by
    \[g(x)=\int_0^x f(t)\,dt=\int_0^x t^{1/3}\,dt=\frac{3}{4}x^{4/3}.\]
    Show that the function \(h(x)=\int_0^x g(t)\,dt\) is \(C^2\) but not \(C^3\) at \(x=0\).
\end{exercise}
\begin{solution}
    The second derivative \(h''(x)=g'(x)=f(x)=x^{1/3}\) is continuous everywhere on its domain, so \(h(x)\) is \(C^2\). The third derivative \(h'''(x)=\frac{1}{3}x^{-2/3}\) has a discontinuity (a vertical asymptote) at \(x=0\); that is, \(h\) is not \(C^3\) at \(x=0\).
\end{solution}

\begin{exercise}
    Show that the function \(f:(-\pi/2,\pi/2)\to\R\), \(f(x)=\tan x\), is a diffeomorphism.
\end{exercise}
\begin{solution}
    First we show that \(f\) is a bijection. A strictly increasing function is injective because it never takes the same value twice. The function \(f\) strictly increasing on its domain since \(f'(x)=\sec^2 x>0\) for all \(x\in(-\pi/2,\pi/2)\); hence \(f\) is injective. Moreover, 
    \[\lim_{x\to-\pi/2^+}=-\infty\quad\text{and}\quad\lim_{x\to\pi/2^-}=\infty.\]
    Therefore, \(f\) maps \((-\pi/2,\pi/2)\) onto the entire real line \(\R\); that is, \(f\) is surjective. 
    
    The function \(f\) is smooth by induction; the first derivative is \(f'(x)=\sec^2 x\), but \(f''(x)=2\tan x\sec^2 x\), so for \(n>2\), \(f^{(n)}\) can be written as the sum of products of differentiable functions, and is therefore itself differentiable. The inverse function \(f^{-1}(y)=\arctan y\) has first derivative \((1+y^2)^{-1}\) with respect to \(y\), which is the reciprocal of a polynomial. For all \(y\in\R\), \(1+y^2\geq1>0\), ensuring that the derivative is well-defined for all real numbers. Thus, \(f^{-1}(y)\) inherits smoothness from the smoothness of polynomials.

    The function \(f\) is a bijective \(C^\infty\) map with \(C^\infty\) inverse, therefore \(f\) satisfies the definition of a diffeomorphism.
\end{solution}

\begin{exercise}
    Find a bijective \(C^\infty\) map \(f:\R\to\R\) that does not have a \(C^\infty\) inverse.
\end{exercise}
\begin{solution}
    Define \(f:\R\to\R\) by \(f(x)=x^3\). The function \(f\) is a polynomial, and hence smooth. Injectivity follows from the fact that \(f\) is strictly increasing on \(\R\), and surjectivity follows from the existence of real cube roots for all \(x\in\R\). Therefore \(f\) is a bijective \(C^\infty\) map.

    The inverse function \(f^{-1}\) is given by \(f^{-1}(y)=y^{1/3}\). The first derivative with respect to \(x\) is given by \(\frac{1}{3}y^{-2/3}\), which is undefined at \(x=0\). Thus \(f^{-1}(y)\) is not differentiable at \(y=0\); it fails to be \(C^1\), let alone \(C^\infty\).
\end{solution}

\subsection*{Day 3}
\begin{exercise}
    Let \(\mathcal{M}\) be an infinite \(\sigma\)-algebra. Show that \(\mathcal{M}\) contains an infinite sequence of disjoint sets.
\end{exercise}
\begin{solution}
    Suppose \(\{E_i\}_{i=1}^\infty\subset\mathcal{M}\). Let
    \[F_j =E_j\setminus \left(\bigcup_{i=1}^{j-1} E_i\right) = E_j \cap \left(\bigcup_{i=1}^{j-1} E_i\right)^\complement.\]
    Then \(\{F_j\}_{j=1}^\infty\) is an infinite sequence of disjoint sets.
\end{solution}

\begin{exercise}
    Find the conditions under which the equation \(az+b\bar{z}+c=0\) has exactly one solution, and compute that solution.
\end{exercise}
\begin{solution}
    We can solve this equation by forming a system of linear equations using the original equation and its conjugate.
    \begin{align*}
        az+b\bar{z}+c&=0 \\
        \overline{az}+\bar{b}z+\bar{c}&=0.
    \end{align*}
    Our goal is to eliminate \(\bar{z}\). We begin by multiplying the first equation by \(\bar{a}\) and the second by \(b\).
    \begin{align*}
        \bar{a}az+\bar{a}b\bar{z}+\bar{a}c&=0 \\
        b\overline{az}+b\bar{b}z+b\bar{c}&=0.
    \end{align*}
    Next, we subtract the second equation from the first to get
    \[\bar{a}az-b\bar{b}z+\bar{a}c-b\bar{c}=0.\]
    Recall that the product of a complex number with its conjugate is the square of its modulus, so we can rewrite our equation as
    \[z=\frac{b\bar{c}-\bar{a}c}{|a|^2-|b|^2}.\]
    Thus, the condition under which the equation admits a solution is that the denominator is nonzero, i.e., \(|a|\neq|b|\).
\end{solution}

\begin{exercise}
    Prove that for any natural numbers \(a,b,c\), we have \((a+b)+c=a+(b+c)\).
\end{exercise}
\begin{solution}
    We use induction on \(a\), keeping \(b\) and \(c\) fixed. First we show \((0+b)+c=0+(b+c)\). By the definition of addition, \(0+b=b\), implying that \((0+b)+c=b+c\). Similarly, \(0+(b+c)=b+c\). This completes the proof of the case \(a=0\). Now suppose inductively that \((a+b)+c=a+(b+c)\). By the definition of addition, 
    \[(S(a)+b)+c=S(a+b)+c\quad\text{and}\quad S(a)+(b+c)=S(a+(b+c)),\]
    where \(S\) is the successor function. The inductive hypothesis gives us that \(S(a+b)+c\) is equal to \(S(a+(b+c))\), as was to be shown. 
\end{solution}

\begin{exercise}
    Let \(E\) be a nonempty subset of an ordered set; suppose \(\alpha\) is a lower bound of \(E\) and \(\beta\) is an upper bound of \(E\). Prove that \(\alpha\leq\beta\).
\end{exercise}
\begin{solution}
    Let \(x\in E\). Then \(\alpha\leq x\) by the definition of lower bound. Similarly, \(x\leq\beta\). By the transitivity of order, \(\alpha\leq\beta\).
\end{solution}

\begin{exercise}
    Let \(A\) be a nonempty set of real numbers which is bounded below. Let \(-A\) be the set of all numbers \(-x\), where \(x\in A\). Prove that \(\inf A=-\sup(-A)\).
\end{exercise}
\begin{solution}
    For any two real numbers \(a\) and \(b\), \(a\leq b\) implies \(-b\leq -a\). Thus, if \(x\in A\) and \(\alpha=\inf A\), then \(-x\leq-\alpha\). Therefore, \(-\alpha\) is an upper bound of \(-A\). Let \(-\beta=\sup(-A)\). Then \(-x\leq-\beta\leq-\alpha\), and, consequently, \(\alpha\leq\beta\leq x\). That is, \(\beta\) is a lower bound of \(A\). But \(\alpha\) is the greatest lower bound of \(A\), and \(\alpha\leq\beta\), so we must have \(\alpha=\beta\). In other words, \(\inf A=-\sup(-A)\).
\end{solution}

\subsection{Day 4}
\begin{exercise}
    Show that \(\Z[i]\) is a subring of \(\C\).
\end{exercise}
\begin{solution}
    Clearly \(\Z[i]\subset\C\). Let \(a,b,c,d\in\Z\). Then addition in \(\Z[i]\) is defined as \((a,b)+(c,d)=(a+c,b+d)\). This operation inherits closure from \(\Z\). Similarly, multiplication in \(\Z[i]\), defined as \((ac-bd,ad+bc)\), inherits closure from \(\Z\). The additive identity \((0,0)\) is in \(\Z[i]\) because \(0\in\Z\). If \((a,b)\in\Z[i]\), then \((-a,-b)\in\Z[i]\) because \(-a,-b\in\Z\). This suffices to show that \(\Z[i]\) is a subring of \(\C\).
\end{solution}

\begin{exercise}
    Let \(a\) and \(b\) be elements of a ring \(R\). Prove that the equation \(a+x=b\) has a unique solution in \(R\).
\end{exercise}
\begin{solution}
    Let \(x_1,x_2\in R\) satisfy \(a+x_1=b\) and \(a+x_2=b\). Then \(x_1=-a+b\) and \(x_2=-a+b\), so \(x_1=x_2\).
\end{solution}

\begin{exercise}
    Let \(S\) and \(T\) be subrings of a ring \(R\). Prove that \(S\cap T\) is a subring of \(R\).
\end{exercise}
\begin{solution}
        Let \(x,y\in S\cap T\). Then \(x,y\in S\). Since \(S\) is a subring, \(0\in S\), \(-x\in S\), \(x+y\in S\), and \(xy\in S\). The same argument applies to \(T\). Hence, each of 0, \(-x\), \(x+y\), and \(xy\) are in \(S\cap T\). Finally, \(S\cap T\subset R\), so \(S\cap T\) is a subring of \(R\).
\end{solution}

\begin{exercise}
    Prove that the only idempotents in an integral domain \(R\) are 0 and 1.
\end{exercise}
\begin{solution}
    Let \(r\in R\), and suppose \(r^2=r\). If \(r^{-1}\) does not exist, then \(r=0\). Otherwise, multiplying both sides by \(r^{-1}\) yields \(r=1\). 
\end{solution}

\begin{exercise}
    If \(R\) is a ring with identity and \(\phi:R\to S\) is a homomorphism from \(R\) to a ring \(S\), prove that \(\phi(1_R)\) is idempotent in \(S\).
\end{exercise}
\begin{solution}
    By the definition of homomorphism, \(\phi(1_R)^2=\phi(1_R^2)=\phi(1_R)\).
\end{solution}

\end{document}
